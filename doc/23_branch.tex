\subsection{Branch instructions}
The implemented instructions are: \texttt{BEQ, BNE, BLT, BGE, BLTU, BGEU}.

\subsubsection{Fetch}
The Fetch stage receives the decision taken if the Memory stage, which indicated if the branch has to be taken or not. 
It also receives the \texttt{pc} address to which branch.
In case, the branch is taken, the \texttt{pc} is updated with the new one received.
If a dependency is found during the Decode stage, the Fetch stage is stalled.

\subsubsection{Decode}
Branch instructions are encoded in the B-type format.
Therefore, signals extracted from the instruction are: \texttt{opcode, func3, rs1, rs2} and \texttt{imm\_se}.

The control type signal set to 1 is \texttt{ctrl\_br}, the other ones are set to 0.
Moreover, the signal \texttt{ctrl\_reg\_write} is set to 0.

\texttt{rs1} and \texttt{rs2} dependencies are checked agains older instructions in the pipeline. 
The dependency behaviour is the same as in \autoref{load_decode} (Decode in Load instructions).

If a branch is taken, the current instruction in the Decode stage is killed.

\subsubsection{Execution}
The condition of the branch is evaluated depending on the \texttt{func3} value.
In other words, the registers are compared according to the instruction condition.
In parallel, the \texttt{pc} address from the branch is calculated.

If a branch is taken, the current instruction in the Execution stage is killed.

\subsubsection{Memory}
To check if the branch has to be taken, the signal \texttt{ctrl\_br} and the signal with the comparison result are checked. 
If both signals are HIGH, then the branch is taken.

\subsubsection{Write back}
No data has to be written into the Register File.



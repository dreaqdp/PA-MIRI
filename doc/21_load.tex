\subsection{Load instructions}
The load instructions implemented are: \texttt{LW, LH, LB}.

\subsubsection{Fetch}
Loads do not affect the usual behaviour of the Fetch stage, which is adding 4 to the \texttt{pc}, unless a dependency is found in the Decode stage.
In that case, the fetch is stalled.

\subsubsection{Decode} \label{load_decode}
Loads are encoded in the I-type format, so the following signals are extracted from the instruction: \texttt{opcode, func3} (width) \texttt{, rs1, imm\_se} and \texttt{rd}.

Moreover, the control signals, which indicate the instruction type, are set to 0 except the \texttt{ctrl\_ld}. That signal shows that the instruction is a load.
Also, the signal \texttt{ctrl\_reg\_write} is set to 1.

\texttt{rs1} and \texttt{rs2} dependencies are checked agains older instructions in the pipeline. 
Both registers are compared agains the \texttt{rd} register from the instructions which are currently in the Execution, Memory and M4 stage.
In case some register matches Execution \texttt{rd} register, the result from the alu is bypassed to the corresponding source register.
The result bypassed from the Memory stage depends on the instruction type that is currently in that stage, if it is a load, then the read memory access data is bypassed, otherwise, the result from the ALU is bypassed.
% In case some register matches Execution or Memory \texttt{rd} register, the result from the read access to memory is bypassed to the corresponding source register.
% Both registers are compared agains the \texttt{rd} register from the instructions which are currently in the Execution and Memory stage.
% In case some register matches Execution or Memory \texttt{rd} register, the result produced from the ALU is bypassed.
If no register \texttt{rd} match, Fetch and Decode are stalled, waiting for the dependency to be solved.

\subsubsection{Execution (ALU)}
In this stage, the load address is computed by adding register \texttt{rs1} and the sign-extended immediate.

\subsubsection{Memory}
The data memory is accessed performing a read to the address computed in the previous stage.
The result sent to the Decode stage is the one from the memory access.

\subsubsection{Write back}
Register \texttt{rd} from the Memory stage and the data from the memory access are selected to be written into the Register File.
